\documentclass[12pt,a4paper]{book}
\usepackage[utf8]{inputenc}
\usepackage[brazil]{babel}
\usepackage{hyperref}
\usepackage{graphicx}
\begin{document}
%% baseado no md5sum 35ea8b85959738edb8f44efb63802f06  oogltour

\vspace*{6cm}
\huge
\begin{center}
Tutorial\\
Os Formatos de Arquivo da OOGL\\
(e Geomview ao longo do caminho)
\end{center}
\normalsize
\vspace*{5cm}
\newpage
\chapter{Introduction}

OOGL significa Object Oriented Graphics Library.  Um objeto OOGL é
chamado um Geom. Existe um formato de arquivo texto para cada tipo de Geom. Você
pode chamar um arquivo texto dentro de qualquer programa que usa a OOGL. Geomview é um
visualizadodr interativo de objetos em 3D construído no topo da OOGL. (Geomview executa nas
plantaformas Silicon Graphics, NextStep, e X-Windows.) Existe
notas do GEOMVIEW em seções específicas distribuídas por esse guia dizendo
a você o que fazer no Geomview para ver sobre o que estamos falando. As seções do GEOMVIEW
subsequêntes assumem que você sabe como fazer e concluiu o que as notas do GEOMVIEW anteriores
disseram sobre Geomview e os formatos de arquivo da OOGL. Todos os arquivos mencionados nesse documento estão no
diretório de arquivos exemplos da OOGL que acompanha o Geomview.

A seção Formatos de Arquivo da OOGL do manual do Geomview é uma referência
completa para a sintaxe de formatos de arquivo. Esse tutorial é uma tentativa
de conduzir você mais gentilmente no mundo da OOGL.

\chapter{QUAD}

Iniciamos com um objeto muito simples: um quadrado. Especificamente, o quadrado
unitário no plano xy e com z=0. O nome do arquivo no diretório do Geomview é ``square.quad'':

\begin{verbatim}
QUAD
-1 -1 0 #vertice 0
1 -1 0  #vertice 1
1 1 0   #vertice 2
-1 1 0  #vertice 3
\end{verbatim}

O cabeçalho ``QUAD'' identifica o tipo de arquivo.  (Você pode também usar o
cabeçalho ``POLY'' para esse tipo por razões históricas.)  Um arquivo QUAD é uma
lista de 4*n vertices onde n é o número de quadriláteros. Esse
arquivo somente contém um quadrilátero.  Você pode também usar esse formato para
especificar triângulos: apenas use um quadrilátero degenerado onde dois dos
quatro vértices são identicos.  Os vértices nesse arquivo são
simples: apenas as coordenadas x,y, e z do ponto.

\section{GEOMVIEW}

Digite ``geomview square.quad'' a partir de uma janela de shell.
Gire o quadrado com o botão esquerdo do mouse
após o geomview carregar o quadrado. Sinta-se livre para brincar com o geomview por um momento se
a imaginação atingir você durante esse tutorial.
\footnote{Nota do tradutor: veja a seção \ref{adc00} para mais informação.}

\section{Colorindo}

O arquivo seguinte tem vértices mais complexos que incluem uma cor no ponto. 
O nome do arquivo no diretório do Geomview é ``csquare.quad'':

\begin{verbatim}
CQUAD
-1 -1 0		1 0 0 1
1 -1 0		0 1 0 1
1 1 0		0 1 0 1
-1 1 0		1 0 0 1
\end{verbatim}

O csquare.quad pegou os mesmos pontos do quadrado anterior, mas com dois cantos vermelhos e
dois cantos verdes. O cabeçalho é agora ``CQUAD'' para indicar que seus
vértices possuem informação de cor além das informações do ponto.
Cores são especificadas por quádruplas (r,g,b,a) de números em ponto flutuante
entre 0 e 1. Qualquer cor que que puder ser mostrada na tela de um computador
pode ser codificada por alguma combinação de vermelho (\textit{red}), verde (\textit{green}) e azul (\textit{blue}). O quarto
componente, alfa, representa opacidade: 0 é transparente e 1 é
opaco. As plantaformas X, NextStep, e algumas plantaformas SGI ignoram a informação
alfa inteiramente, mas um instantâneo Renderman irá usar a informação
alfa se a transparência estiver habilitada. Algumas plantaformas SGI usan a
informação alfa, mas a figura é garantidamente incorreta.

\section{GEOMVIEW}

Apague o objeto atual pressionando o Botão Delete.  Para
chamar csquare.quad, pressione o Botão Load e digite ``csquare.quad'' (pressione
Enter quando você tiver terminado a digitação) dentro da janela que aparece. (Se
você está procurando e não tem certeza no nome dos arquivos, você pode usar o
Botão File Browser (navegador de arquivos) para olhar em um diretório.)

Você está provavelmente maravilhado porque está tudo em uma cor.  O modo de sombreamento
padrão é monótono, onde cada polígono ou face poligonal é da mesma
cor. Os outros dois modos de sombreamento, constante e suave, irão ambos
mostrar faces multicoloridas onde as cores suavemente interpolada
entre os vértices. Sombreamento Constant ignora todas as informações de iluminação,
enquanto sombreamento smooth (suave) interpola iluminação bem como cores entre
os vértices.

Para mudar o modo de sombreamento, primeiro abra o painel Appearance clicando
na linha Appearance no menu Inspect.  Agora alterne de
modo para modo clicando sobre diferentes linhas no navegador de sombreamento.
\footnote{Nota do tradutor: CSmooth - suave com mais brilho, VCflat - monótono com menos brilho}

Hora de ir para coisas maiores e melhores. 

\section{DODEC.QUAD}

\begin{verbatim}
"dodec.quad":

QUAD
0.467086 0.151765 0.794654 0.356822 0.491123 0.794654 0 0.491123 0.794654 0 0 0.794654
4.89153e-09 0.491123 0.794654 -0.356822 0.491123 0.794654 -0.467086 0.151765 0.794654 0 0 0.794654
-0.467086 0.151766 0.794654 -0.57735 -0.187593 0.794654 -0.288675 -0.397327 0.794654 0 0 0.794654
-0.288675 -0.397327 0.794654 4.36694e-09 -0.607062 0.794654 0.288675 -0.397327 0.794654 0 0 0.794654
0.288675 -0.397327 0.794654 0.57735 -0.187592 0.794654 0.467086 0.151766 0.794654 0 0 0.794654
-0.467086 0.642889 0.491123 -0.356822 0.491123 0.794654 0 0.491123 0.794654 0 0.710761 0.35538
	.
	.
	.
 < 53 lines of numbers deleted>
\end{verbatim}
