\documentclass[12pt,a4paper]{book}
\usepackage[utf8]{inputenc}
\usepackage[brazil]{babel}
\usepackage{hyperref}
\usepackage{graphicx}
\begin{document}
%% baseado no md5sum 35ea8b85959738edb8f44efb63802f06  oogltour

\vspace*{6cm}
\huge
\begin{center}
Tutorial\\
Os Formatos de Arquivo da OOGL\\
(e Geomview ao longo do caminho)
\end{center}
\normalsize
\vspace*{5cm}
\newpage
\chapter{Introduction}

OOGL significa Object Oriented Graphics Library.  Um objeto OOGL é
chamado um Geom. Existe um formato de arquivo texto para cada tipo de Geom. Você
pode chamar um arquivo texto dentro de qualquer programa que usa a OOGL. Geomview é um
visualizadodr interativo de objetos em 3D construído no topo da OOGL. (Geomview executa nas
plantaformas Silicon Graphics, NextStep, e X-Windows.) Existe
notas do GEOMVIEW em seções específicas distribuídas por esse guia dizendo
a você o que fazer no Geomview para ver sobre o que estamos falando. As seções do GEOMVIEW
subsequêntes assumem que você sabe como fazer e concluiu o que as notas do GEOMVIEW anteriores
disseram sobre Geomview e os formatos de arquivo da OOGL. Todos os arquivos mencionados nesse documento estão no
diretório de arquivos exemplos da OOGL que acompanha o Geomview.

A seção Formatos de Arquivo da OOGL do manual do Geomview é uma referência
completa para a sintaxe de formatos de arquivo. Esse tutorial é uma tentativa
de conduzir você mais gentilmente no mundo da OOGL.
