%
% $Header: /home/mbp/geomview-git/geomview-cvs/geomview/src/bin/example/tutorial.tex,v 1.1 2000/08/15 16:32:29 mphillips Exp $
% $Source: /home/mbp/geomview-git/geomview-cvs/geomview/src/bin/example/tutorial.tex,v $

% This is a LaTeX document.  To print it out, enter the following
% commands from the shell: 
%	
%	latex explain
%	lpr -d explain.dvi
%

% $Log: tutorial.tex,v $
% Revision 1.1  2000/08/15 16:32:29  mphillips
% Initial revision
%
% Revision 1.1  2000/06/06 18:59:51  mbp
% *** empty log message ***
%
% Revision 1.3  1992/09/22  20:54:26  munzner
% expand makefile explanation.
% fix typos pointed out by Oliver.
%
% Revision 1.2  1992/09/04  23:46:40  mbp
% change ui-emotion-program to emodule-define
%
% Revision 1.1  1992/09/04  23:30:20  gunn
% Initial revision
%
% Revision 1.9  90/04/13  09:17:20  meuer
% First revision after Charlie has added stuff to it.
% 
% Revision 1.8  90/04/13  09:10:19  meuer
% Draft with Charlie's corrections
% 


\documentstyle[cprog]{article}
%\documentstyle[cprog,doublespace]{article}

\newcommand{\progname}[1]{{\tt #1}}
\newcommand{\filename}[1]{{\tt #1}}
\newcommand{\modulename}[1]{{\tt #1}}
\newcommand{\OOGL}{{\em OOGL}}

\title{A Guide to Using \\
\progname{geomview} and {\OOGL}}

\date{ $ $Revision: 1.1 $ $ \\
$ $Date: 2000/08/15 16:32:29 $ $}


\author{Charlie Gunn}

\begin{document}

\maketitle

\section{Introduction}

This document will show you how to write programs using the Object
Oriented Graphics Library ({\OOGL}).  It will also give you an introduction
to using \progname{geomview}---a 3-dimensional viewing program
written using {\OOGL}.  By using \progname{geomview}, your application
is freed from  having to do any specific graphics functions.
\footnote{References to files in this document will use the
symbol GEOM to refer to the directory where the geomview
distribution has been installed on your system.}

The {\OOGL} routines are a high-level set of modules that let you specify
and display geometry.  There are several types of geometric
primitives which OOGL can handle, including polygons, vectors,
meshes, and parametric surfaces patches.  In addition, these primitives
can be organized into a higher-order lists.  
Also, multiple instances of the same geometry can be
created by specifying different modeling transformations.
Each sort of geometric object, whether low-level or high-level,
comes equipped with a standard set of operations.
Examples of these operations include 
loading from and saving to files, 
drawing to a display device,
and computing a bounding box.
This style of programming, which combines data structures
with operations which act on these data structures, is known as
object-oriented programming.
Although currently {\OOGL} only works on SGI workstations, it
has been designed to make it easy to port to other environments.

It is assumed you know how to program in C.  After reading this and
looking through the example programs you will be ready to create your
own {\OOGL} applications.

\section{Integrating your application and and \progname{geomview}}

\subsection{Why Use \progname{geomview}?}

\progname{geomview} is a program for the display and
manipulation of {\OOGL} objects.  
If you haven't already done so, you should familiarize yourself
with its capabililities now. Consult the documents  
\filename{overview} and  \filename{oogltour} in the
\filename{doc} directory, and the man page
for \progname{geomview}, both sections 1 and 5.
The section 1 manual page describes the user interface
of the program, while the section 5 manual page describes
the command language through which external programs
can communicate with \progname{geomview}.   It is this
command language which this example will use.

This corresponds to 2 different ways to provide 
\progname{geomview} 
with geometry for viewing.  The first reads descriptions of objects
from a file. \progname{geomview} has a file browser from 
which users can choose files to examine.
This is appropriate when the geometry is static.  The
second method is useful when you desire to view a changing geometry.
This second method uses 2-way pipes to update the changing geometry
from your application
and to pass back information from \progname{geomview} to your application.
This communication uses the command language described in
\progname{geomview(5)}.
In this situation, your application is known as an {\em external
module} for \progname{geomview}.

This document will show how to create and run an external module,
in this case one simply known as \progname{example}.
This and all the other files involved in this are to be found
in the directory \filename{GEOM/src/bin/example}.

\section{How to Write an External Module}

\subsection{Structure of the \OOGL\  Routines}

\OOGL\  stands for ``Object-Oriented Graphics Library.''  
The library
defines a class of ``geometry objects.''  As indicated above,
these are object-oriented 
in the sense that a geometry object is a
data structure plus a set of procedures, often called
``methods'', which operate on the data.  
The object is accessible only through its methods.

Another feature of object-oriented programming is the capacity
to create subclasses of a class.
There are a number of subclasses of the generic geometry  class Geom.
In practice, each instance of the Geom class is also an instance  
of a specific subclass, but this isn't necessarily true in general. 
Example subclasses include: vectors, polygons,
meshes, parametric surface patches, and bounding boxes.
There are also composite subclasses built on these, such as
lists and instances.  See oogl(5) for a description of the
file formats belonging to these subclasses.  

All subclasses of geometry objects have a set of standard methods used to
manipulate the objects.  Some example methods are: Create, Delete,
Draw, Load, Save, and Print.  
The name of a subclass's methods
begin with the name of that subclass.
For example, the Draw method for a mesh is named MeshDraw.
In general a subclass can inherit
methods from its superclasses; but in {\OOGL} almost all subclass
methods are provided specifically for that subclass.
These methods behave differently
because they are each customized to the particular subclass of 
geometry object to which they belong.  
The details of the data structures and existing methods
for each subclass can be found in the include file for that
subclass. These are kept in
\filename{GEOM/include}.

\subsubsection{The Generic Class Geom}
There is a type of programming task which is made much easier by the fact
that each instance of a subclass is also an instance of
the generic class. In our case, each instance of a geometry subclass is
also an instance of the generic Geom class.
So if our program, like \progname{geomview}, is designed to
deal with many different types of geometry objects, we can simplify
the program structure by always working with the Geom class rather
than the specific subclasses.  GeomLoad and GeomDraw, for example,
work perfectly well to load and draw instances of any subclass, too.
Of course it takes some work to make sure this works, but that work
has already been done for you.

\subsection{\progname{example}: A Minimal {\OOGL} Program}

\progname{example} 
demonstrates the use of the mesh {\OOGL} object. 
It repeatedly computes a changing function on a
rectangular grid, or mesh, of $(x,y)$ pairs.  
As the mesh is changed by
\progname{example} over time, these changes dynamically appear in the
\progname{geomview} window.  

A second feature of \progname{example} is the use of the Forms
interface builder to create the user interface.  This is a public
domain interface builder which has been used to create 
\progname{geomview}.  It is included here because we have found
it to be an effective tool for creating interactive applications.
The Forms package is written by Mark Overmars of
Utrecht University and is available via anonymous ftp from archive.cs.ruu.nl.

\subsubsection{Preparing to run \progname{example}}

First, check that \filename{GEOM/bin/sgi/example} exists.
If not, first change directory to \filename{GEOM/src/bin/example}
and type

\begin{verbatim}
%  make install
\end{verbatim}

You'll next need to be sure that \progname{example} is
registered with \progname{geomview}.  To do this, edit the
file \filename{.geomview} in your home directory and add
the line

\begin{verbatim}
  (emodule-define "Example" "example")
\end{verbatim}

This file will be read by \progname{geomview} when it starts
up and the line you've added tells it to register the
external module \progname{example} to its list of external modules,
under the name "Example".  As long as \progname{example} exists
in \filename{GEOM/bin/sgi}, \progname{geomview} will find it there.
Or, you can provide a full pathname for your application, which
will guarantee that it will be found.  Also, if your application
has command line arguments, include them here.

Then, run \progname{geomview}.  On the main panel, under the
"Modules",  "Example" should appear.  Pick this entry.  It 
should be highlighted in yellow, and  you should be prompted to
place a small panel on the screen.  This is the user-interface panel
create using the Forms designer.  Soon after you place this panel,
you should see an object appear in the geomview viewing window.
This is the output from the \progname{example} module.  As you
watch, the geometry should change.  If you want this change to be
slower or faster, you can move the single slider in the panel
which controls the 'velocity' of the change.  

\subsubsection{The \progname{example} Code}

The file \filename{main.c} contains all of the non-{\OOGL} code.  The
routine \modulename{myfunc} is the function to be computed over the
mesh.  Its arguments are the $x$ and $y$ location on the mesh and
the time since the start of the program.

The main routine first performs a number of initialization steps.
The call to \modulename{Begin\_OOGL} essentially creates an empty 
geometry template and returns:

\begin{verbatim}
fprintf(f, "(geometry example { : exhandle })\n");
fflush(f);
\end{verbatim}

This notifies the viewer to create an object known as 
{\em example}.  It gives it the internal name, or handle, 
{\em exhandle}. 
More on this later when we come to \modulename{UpdateOOGL}

After initializing everything, \modulename{main} enters an endless
loop where it computes the values of the function over each point in
the mesh and then calls \modulename{UpdateOOGL} to update the {\OOGL}
mesh in shared memory.  User interface allows the user to change
the value of {\em dt}. The resulting data array and other parameters
are passed to \modulename{UpdateOOGL}. 

\modulename{UpdateOOGL}'s main task is to create a Mesh object
from the data which has been passed in.  This is the one slightly
tricky section of the program, since the particular features of
each subclass must be known in order to create an instance.

All the Create routines are accessed through the generic 
\modulename{GeomCreate} routine.  The first parameter is an ascii
string identifying the subclass, in this case "mesh".  From then
on, the Create routine expects a sequence of keys 
accompanied with optional key values.  A complete list of these
keys is given in \filename{GEOM/include/create.h}.  There are a
number of generic sorts of keys, such as CR\_NOCOPY or CR\_POINT, which are valid
for any subclass.  CR\_NOCOPY tells the create routines to
use the data you have provided without copying it over -- so we
are careful not to delete the data storage we are passing.
CR\_POINT identifies an array of 3-dimensional vertices; CR\_POINT4
does the same for 4-dimensional vertices.

Other of these keys, however, are specific to
the Mesh subclass.  For example, CR\_NU and CR\_NV specify the
dimensions of the mesh.  Last and most difficult, CR\_FLAG identifies
a set of flag values which vary with each subclass and can
only be fully learned by consulting the specific subclass.h file.
In this case, we consult \filename{meshflag.h} to  find that
we must use the MESH\_Z flag  for our data, since we aren't 
providing explicit (x,y) data with the z-values.

The best way to  become comfortable with this aspect of the 
Create routines is to examine other examples in the \filename{GEOM/src/bin}
directory, and explore the include files corresponding to your
favorite subclass.  

\modulename{GeomCreate} returns a pointer to a Geom upon successful
completion.  It also happens to be a pointer to a Mesh, but we
don't care any more what kind of Geom it is.  

The final step is to put this mesh out onto the pipe.  We do this
by wrapping a call to \modulename{GeomFSave} within  a simple
command which connects this geometry to the afore-mentioned
handle {\em exhandle}.  \modulename{GeomFSave} prints onto
the given file stream, in this case, standard output,
the value of the Geom.  Then, when \progname{geomview} reads
this stream, it will redefine {\em exhandle} to replace any
old definitions.

\section{Remarks}

It is a good idea to use the \modulename{OOGLNew} routines and
their variants for allocating storage and \modulename{OOGLFree} for
freeing it.  See \filename{GEOM/include/ooglutil.h} for details.

A good way to debug external modules is to run them stand-alone
first, and examine the ascii stream which they produce to
check for obvious problems.  Or catch it this stream in a file, edit the
file, and then read that into \progname{geomview} to see if
it's making reasonable pictures.

The Makefile mechanism illustrated in this example is quite complex.
The basic structure is that the source code exists at the
\filename{GEOM/src/bin} level, but there are multiple object
directories corresponding to possibly different architectures (in our
case \filename{O.sgi} is the one of interest).  The parent Makefile
then descends into that directory and does the heart of the make.  All
the Makefiles include global makefiles from \filename{GEOM/makefiles},
which define many macros that you don't need to understand. The
important ones are explained below.  There is a file called
\filename{Makedefs} in the parent directory that defines the macros
SRCS, OBJS, and TARGET, which you will need to set to the appropriate
filenames. (The DISTFILES macro can be safely ignored.) The only
change you should need to make to the \filename{Makefile} in the
parent directory is to set the GEOM variable to the appropriate
directory, which is the top of the geomview tree. This change also
should be made to \filename{O.sgi/Makefile}, which you can then modify
as appropriate for your application. Remember that after modifying any
of the makefiles you should type 
\begin{verbatim} 
% make depend
\end{verbatim}
Enjoy.

\section{Listings}

These are the listings of all the demo code mentioned in this article.

\subsection{example}
\label{ExampleListings}
\subsubsection{File: {\tt main.c}}

\cprogfile{main.c}

\subsubsection{File: {\tt oogl.c}}

\cprogfile{oogl.c}

\subsubsection{File: {\tt Makedefs}}
\begin{verbatim}
SRCS = main.c oogl.c callbacks.c
OBJS = main.o oogl.o callbacks.o
TARGETS = example
DISTFILES = panel.fd panel.c tutorial.tex tutorial.ps README
\end{verbatim}

\subsubsection{File: {\tt Makefile}}
\begin{verbatim}
GEOM = ../../..
include ${GEOM}/makefiles/Makedefs.global
include Makedefs
include ${GEOM}/makefiles/Makerules.src
\end{verbatim}

\subsubsection{File: {\tt O.sgi/Makefile}}
\begin{verbatim}
GEOM = ../../../..
include ${GEOM}/makefiles/Makedefs.global
include ../Makedefs
include ${GEOM}/makefiles/Makerules.obj

FORMSLIBS = -lforms -lfm_s
ALLLIBS = ${ALLOOGLLIBS} ${FORMSLIBS} -lgl_s -lm

example:	${OBJS} 
	rm -f example ../example
	cc ${CFLAGS} ${OBJS}  ${ALLLIBS} -o example
	ln example ..

tutorial.dvi:	tutorial.tex
	latex tutorial

install:	install_bin

install_bin:	example
	${INSTALL} -O -v -F ${GEOM}/bin/${MACHTYPE} example
	strip ${GEOM}/bin/${MACHTYPE}/example
	chmod 555 ${GEOM}/bin/${MACHTYPE}/example
\end{verbatim}

\end{document}


